\sectionnonum{Įvadas}
Pasirinkimą pradėti karjerą \enquote{Bentley Systems} nulėmė įmonės dydis, ilgaamžiškumas ir pagrindinis dėmesys programinės įrangos sprendimams. Šie aspektai buvo svarbūs, nes garantuoja, kad įmonėje kreipiamas dėmesys ne tik į galutinį produktą, bet ir kodavimo standartų, geriausios praktikos laikymąsi. Taip pat įmonės dydis garantuoja, kad joje dirbantys darbuotojai yra patyrę, savo srities ekspertai, kurie padės išspręsti iškilusias problemas ir įgyti darbo su dideliu produktu, darbo komandoje patirties. Įmonė didelį dėmesį skiria ir naujų profesionalų parengimui. Gerai pasirodę praktikantai, kviečiami dalyvauti 2 metų rotacijos programoje, kur kiekvienos rotacijos metu inžinieriai skatinami pasirinkti įvairias programavimo technologijas, siekiant rasti jam labiausiai patikusią komandą ir sritį.
Kiekviena rotacija šiose komandose trunka pusmetį ir programuotojams leidžia  susipažinti įmonės įvairiais produktais, komandos nariais. Baigus šias rotacijas, programuotojas gali pasirinkti iš komandų, kurioje toliau tęs savo karjerą.
\bigskip

Šios praktikos \textbf{tikslas} - mobiliosios programėlės funkcionalumo kūrimas ir tobulinimas.
\bigskip

Šios praktikos \textbf{uždaviniai}:
\begin{enumerate}
    \item Pagilinti Kotlin programavimo kalbos žinias rašant programinį kodą.
    \item Išmokti naudotis “Jetpack Compose” Android įrankių rinkiniu skirtu kurti vartotojo sąsają.
    \item Susipažinti su lanksčiojo programavimo projektų valdymo metodu “Kanban”.
    \item Susipažinti su švaraus ir kokybiško kodo rašymo principais ir geriausiomis praktikomis. 
    \item Įgyvendinti naują funkcionalumą Android programėlėje.
\end{enumerate}
\bigskip

Praktika truko 10-11 savaičių, ją atlikau 2024-02-05 -- 2022-04-15.
\bigskip
Šio darbo pirmajame skyriuje aprašoma įmonė \enquote{Bentley systems}, jos veiklos sritis, organizacinė struktūra ir sudarytos darbo sąlygos. Antrajame skyriuje aprašoma praktikos veikla -- tai, kokia buvo užduotis ir kaip ji įgyvendinta. Trečiajame skyriuje pateikiami praktikos darbo rezultatai ir išvados, privalumai ir trūkumai, įgytos žinios bei rekomendacijos universitetui ir organizacijai.
