\documentclass{VUMIFPSPraktika}
\usepackage{float}
\usepackage{hyperref}
\usepackage{algorithmicx}
\usepackage{algorithm}
\usepackage{algpseudocode}
\usepackage{amsfonts}
\usepackage{amsmath}
\usepackage{bm}
\usepackage{caption}
\usepackage{color}
\usepackage{graphicx}
\usepackage{listings}
\usepackage{subcaption}
\usepackage{wrapfig}
\usepackage{biblatex}
\usepackage{microtype}
\usepackage{xcolor}
\usepackage{booktabs}
\usepackage{pgfplots}
\usepackage{multirow}

% Kableliams skaičiuose
\usepackage{icomma}
\usepackage{siunitx}
\sisetup{output-decimal-marker={,}}

\pgfplotsset{compat=newest}

\bibliography{bibliografija}

\paper{Praktikos ataskaita}
\author{Pijus Petkevičius}
\title{Mobiliosios programėlės funkcionalumo kūrimas ir tobulinimas}
\englishtitle{Mobile application functionality implementation and refinement}
\department{Programų sistemų bakalauro studijų programa}
\universitysupervisor{Lect. Irus Grinis}
\organisationsupervisor{\parbox{\linewidth}{\raggedright
Vyr. programinės įrangos inžinierius Vytautas Juozas Barkauskas}}
\date{Vilnius, \the\year}

\begin{document}

\maketitle––≠≠
\tableofcontents

\sectionnonum{Įvadas}
% Įvadas. Išdėstomi praktikos vietos pasirinkimo motyvai, praktikos užduotis, jos tikslas, spręstieji uždaviniai, pateikiama praktinės veiklos planas praktikos atlikimo eiga (2-3 psl.).
Praktiką atlikau įmonėje \enquote{Bentley systems}. Šią įmonę pasirinkau todėl, kad joje organizuojama rotacijų programa, kuri leidžia išbandyti įvairias technologijas, kiekvienos rotacijos metu. Programinės įrangos inžinieriai skatinami rinktis kuo įvairesnes technologijas, kad pasibaigus rotacijos periodui, inžinierius galėtų dirbti jam labiausiai patikusioje komandoje su mėgstamomis technologijomis.
\bigskip

Šios praktikos \textbf{tikslas} - mobiliosios programėlės funkcionalumo kūrimas ir tobulinimas.
\bigskip

Šios praktikos \textbf{uždaviniai}:
\begin{enumerate}
    \item Pagilinti Kotlin programavimo kalbos žinias rašant programinį kodą.
    \item Išmokti naudotis “Jetpack Compose” Android įrankių rinkiniu skirtu kurti vartotojo sąsają.
    \item Susipažinti su lanksčiojo programavimo projektų valdymo metodu “Kanban”.
    \item Susipažinti su švaraus ir kokybiško kodo rašymo principais ir geriausiomis praktikomis. 
    \item Įgyvendinti naują funkcionalumą Android programėlėje.
\end{enumerate}
\bigskip

Praktika truko 10-11 savaičių, ją atlikau 2024-02-05 -- 2022-04-15.

Šio darbo pirmajame skyriuje aprašoma įmonė \enquote{Bentley systems}, jos veiklos sritis, organizacinė struktūra ir sudarytos darbo sąlygos. Antrajame skyriuje aprašoma praktikos veikla -- tai, kokia buvo užduotis ir kaip ji įgyvendinta. Trečiajame skyriuje pateikiami praktikos darbo rezultatai ir išvados, privalumai ir trūkumai, įgytos žinios bei rekomendacijos universitetui ir organizacijai.

\section{Įmonės apibūdinimas}
Šiame skyriuje aprašoma įmonė \enquote{Bentley systems}, jos veiklos sritis, organizacinė struktūra ir sudarytos darbo sąlygos.
% Įmonės/įstaigos apibūdinimas. Glaustai aprašoma įmonė/įstaiga, kurioje buvo atlikta praktika: jos veiklos sritis, organizacinė struktūra, teikiamos paslaugos ir kt. Apibūdinamos praktikos vietoje sudarytos darbo sąlygos (1-2 psl.).
\subsection{Įmonės veiklos sritis}
\enquote{Bentley systems} įsikūrė 1984 metais 
\subsection{Įmonės organizacinė struktūra}
\subsection{Įmonės sudarytos darbo sąlygos}

\section{Praktikos veiklos aprašymas}
% Praktikos veiklos aprašymas (vienas arba keli skyriai). Aprašomas praktikos užduoties įgyvendinimas (pvz., atlikti projektavimo ir/ar programavimo darbai, sukurtas modelis, priimti sprendimai ir pan.).
\subsection{Praktikos veiklos aprašymas}
\subsection{Praktikos užduoties įgyvendinimas}

\section{Rezultatai, išvados ir pasiūlymai}
% Rezultatai, išvados ir pasiūlymai. Išdėstomi pagrindiniai darbo rezultatai ir išvados, praktikos darbo privalumai ir trūkumai, aprašomos įgytos žinios ir patirtis praktikos metu, duodamas universitete įgytų žinių atitikimo praktikos užduočiai atlikti įvertinimas, pateikiami argumentuoti pasiūlymai, kaip geriau organizuoti darbo ir valdymo procesus praktikos atlikimo vietoje ir mokymą Universitete (1-2 psl.).
\subsection{Darbo rezultatai ir išvados}
\subsection{Praktikos darbo privalumai ir trūkumai}
\subsection{Įgytos žinios ir patirtis praktikos metu}
Įgyta patirtis, kompetencijos ir gebėjimai:
\begin{itemize}
    \item Įgyta patirties rašant programinį kodą Kotlin programavimo kalba.
    \item Įgyta žinių apie “Jetpack Compose” Android įrankių rinkiniu skirtu kurti vartotojo sąsają.
    \item Lavinau darbo komandoje gebėjimus ir darbo "Kanban" projektų valdymo metodo pagalba.
    \item Įgyta patirties rašant kodą su švaraus ir kokybiškorašymo principais ir geriausiomis praktikomis. 
    \item Įgyvendintas naujas funkcionalumas Android programėlėje.
\end{itemize}

\subsection{Pasiūlymai įmonei}
\subsection{Pasiūlymai universitetui}

\end{document}
